% tempalte.tex
% Author: 傅申
% 本文件是 实验报告模板.docx 与 实验报告模板.tex 的衍生作品,其著作权属于中国科学技术大学计算机实验教学中心.

% 本文件可使用TeXLive发行版(作者使用的是TeXLive 2021,但理论上2016及以后均可)的xelatex命令编译.
\documentclass[UTF8,fontset=fandol]{ctexart}
\usepackage{LabReport}

\begin{document}
\maketitle{使用 Vivado 进行仿真}
\section*{实验目的}
%列举本实验的实验目的,除指导书上列出的之外,鼓励自行总结及扩展。
\begin{itemize}
    \item 熟悉 Vivado 软件的下载、安装及使用
    \item 学习使用 Verilog 编写仿真文件
    \item 学习使用 Verilog 进行仿真,查看并分析波形文件
\end{itemize}
\section*{实验环境}
%列举本实验所用到的各种软硬件环境,如EDA工具、实验平台、实验设备等。
\begin{itemize}
    \item Windows PC
    \item Microsoft Visual Studio Code
    \item Xilinx Design Tools Vivado HL Design Edition 2019.1 
\end{itemize}
\section*{实验练习}
%如无特殊说明,则应完成对应实验手册上的所有练习题目,将过程和结果以图文并茂的形式体现在本报告中,建议实验过程中随手保存各种截图。
\begin{ExQuestions}
  \question 编写的 Verilog 仿真文件如下
  \begin{lstlisting}[style=verilogstyle, caption={题目 1 的仿真文件}, label={Code.1}]
`timescale 1ns / 1ps
module test_bench();
reg  a, b;

initial #400 $finish;

initial 
begin
    a = 1'b1;
    #200 a = 1'b0;
end

initial 
begin
    b = 1'b0;
    #100 b = 1'b1;
    #175 b = 1'b0;
    #75  b = 1'b1;
end
endmodule
  \end{lstlisting}
  生成的仿真波形如下图 \ref{Fig.1}.
  \img{1}{images/fig1}{题目 1 的仿真波形}{Fig.1}

  \question 编写的 Verilog 仿真文件如下
    \begin{lstlisting}[style=verilogstyle, caption={题目 2 的仿真文件}, label={Code.2}]
`timescale 1ns / 1ps
module test_bench();
reg clk, rst_n, d;

initial #55 $finish;

initial clk = 1'b0;
always #5 clk = ~clk;

initial
begin
    rst_n = 1'b0;
    #27 rst_n = 1'b1;
end

initial 
begin
    d = 1'b0;
    #13 d = 1'b1;
    #24 d = 1'b0;
end
endmodule
    \end{lstlisting}
    生成的仿真波形如下图 \ref{Fig.2}.
    \img{1}{images/fig2}{题目 2 的仿真波形}{Fig.2}

    \question 编写的 Verilog 仿真文件如下
    \begin{lstlisting}[style=verilogstyle, caption={题目 3 的仿真文件}, label={Code.3}]
`timescale 1ns / 1ps
module test_bench();
reg  clk, rst_n, d;
wire q;

d_ff_r d_ff_r(.clk(clk),.rst_n(rst_n),.d(d),.q(q));

initial #55 $finish;

initial clk = 0;
always #5 clk = ~clk;

initial
begin
    rst_n = 0;
    #27 rst_n = 1;
end

initial 
begin
    d = 0;
    #13 d = 1;
    #24 d = 0;
end

endmodule
    \end{lstlisting}
    生成的仿真波形如下图 \ref{Fig.3}.
    \img{1}{images/fig3}{题目 3 的仿真波形}{Fig.3}

    \question 设计的 3-8 译码器模块的代码如下
    \begin{lstlisting}[style=verilogstyle, caption={3-8 译码器模块}, label={Code.4}]
// 模块文件: Design Sources/decoder_3to8.v
module decoder_3to8(
    input  [3:0] in,
    output reg a, b, c, d, e, f, g, h
);
always @(*) 
begin
    a = 0;
    b = 0;
    c = 0;
    d = 0;
    e = 0;
    f = 0;
    g = 0;
    h = 0;
    case(in)
        3'b000: a = 1;
        3'b001: b = 1;
        3'b010: c = 1;
        3'b011: d = 1;
        3'b100: e = 1;
        3'b101: f = 1;
        3'b110: g = 1;
        3'b111: h = 1;
        default:;
    endcase
end
endmodule
    \end{lstlisting}
    仿真测试文件如下
    \begin{lstlisting}[style=verilogstyle, caption={题目 4 仿真测试文件}, label={Code.5}]
// 仿真文件: Simulation Sources/test_bench.v
`timescale 1ns / 1ps
module test_bench();
reg  [3:0] in;
wire a, b, c, d, e, f, g, h;
decoder_3to8 decoder(.in(in), .a(a), .b(b), .c(c), .d(d),
                     .e(e), .f(f), .g(g), .h(h));

initial 
begin
    in = 3'b000;
    repeat(7) #10 in = in + 1;
    #10 $finish;
end
endmodule
    \end{lstlisting}
    生成的仿真波形如下图 \ref{Fig.4}.
    \img{1}{images/fig4}{题目 4 的仿真波形}{Fig.4}
\end{ExQuestions}
\section*{总结与思考}
%请填写对于本次实验的总结与思考,鼓励填写对于本实验或者本课程的各种建议和吐槽。
\begin{description}
    \item[收获] 学习了如何编写仿真文件并使用 Vivado 进行仿真
    \item[难易程度] 简单
    \item[任务量] 轻松
    \item[建议] 第 2 题与第 3 题可以合并 
\end{description}
\end{document}
